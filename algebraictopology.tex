\documentclass[twocolumn]{article}
\usepackage[a4paper, total={7.5in, 10.7in}]{geometry}
\usepackage{amsthm}
\usepackage{amssymb}
\usepackage{enumitem}

\theoremstyle{definition}
\newtheorem{definition}{Definition}[section]
\newtheorem{theorem}[definition]{Theorem}
\newtheorem{lemma}[definition]{Lemma}
\newtheorem{proposition}[definition]{Proposition}

\begin{document}

\section{Algebraic Topology}
\begin{definition}[de Rham complex]
    $\Omega^*$ is the algebra generated over $\mathbb{R}$ by $dx_1, \dots, dx_n$ subject to
    \setlist{nolistsep}
    \begin{enumerate}[noitemsep]
        \item $(dx_i)^2 = 0$,
        \item $dx_i dx_j = -dx_j dx_i, i \neq j$.
    \end{enumerate}
    The $C^\infty$ differential forms on $\mathbb{R}$ are elements of
    \begin{equation}
        \Omega^*(\mathbb{R}^n) = \{ C^\infty \textrm{ functions on } \mathbb{R}^n \} \otimes_\mathbb{R} \Omega^*.
    \end{equation}
    We have $\Omega^*(\mathbb{R}^n) = \oplus^n_{q=0} \Omega^q(\mathbb{R}^n)$,
    where $\Omega^q(\mathbb{R}^n)$ consists of the $C^\infty$ $q$-forms on $\mathbb{R}^n$. We define
    \begin{equation}
        d : \Omega^q(\mathbb{R}^n) \rightarrow \Omega^{q+1}(\mathbb{R}^n),
    \end{equation}
    the exterior differentiation, by
    \setlist{nolistsep}
    \begin{enumerate}[noitemsep]
        \item if $f \in \Omega^0(\mathbb{R}^n)$, then $df = \sum \partial f / \partial x_i \, dx_i$,
        \item if $\omega = \sum f_I dx_I$, then $d\omega = \sum df_I dx_I$, where $dx_I = dx_i dx_j \dots $.
    \end{enumerate}
    The wedge product is defined by
    \begin{equation}
        \tau \wedge \omega = \sum \tau_I \omega_J \, dx_I dx_J = (-1)^{\mathrm{deg }\tau\,\mathrm{deg }\omega} \omega \wedge \tau.
    \end{equation}
\end{definition}
\begin{proposition}
    $d$ is an antiderivation,
    \begin{equation}
        d(\tau \wedge \omega) = (d\tau) \wedge \omega + (-1)^{\mathrm{deg} \tau} \tau \wedge d\omega.
    \end{equation}
\end{proposition}
\begin{proposition}
    $d^2 = 0$.
\end{proposition}
\begin{definition}
    The $q$-th de Rham cohomology of $\mathbb{R}^n$ is the vector space
    \begin{equation}
        H^q(\mathbb{R}^n) = \{\textrm{closed $q$-forms}\}/\{\textrm{exact $q$-forms}\},
    \end{equation}
    where closed means in the kernel of $d$ and exact means in the image of $d$.
    We denote by $[\omega]$ the cohomology class of $\omega$.
\end{definition}

\end{document}