\documentclass[twocolumn]{article}
\usepackage[a4paper, total={7.5in, 10.7in}]{geometry}
\usepackage{amsthm}
\usepackage{amssymb}
\usepackage{enumitem}
\usepackage{tikz-cd}

\theoremstyle{definition}
\newtheorem{definition}{Definition}[section]
\newtheorem{theorem}[definition]{Theorem}
\newtheorem{lemma}[definition]{Lemma}
\newtheorem{proposition}[definition]{Proposition}

\begin{document}

\section{Algebraic Topology}
\begin{definition}[de Rham complex]
    $\Omega^*$ is the algebra generated over $\mathbb{R}$ by $dx_1, \dots, dx_n$ subject to
    \setlist{nolistsep}
    \begin{enumerate}[noitemsep]
        \item $(dx_i)^2 = 0$,
        \item $dx_i dx_j = -dx_j dx_i, i \neq j$.
    \end{enumerate}
    The $C^\infty$ differential forms on $\mathbb{R}$ are elements of
    \begin{equation}
        \Omega^*(\mathbb{R}^n) = \{ C^\infty \textrm{ functions on } \mathbb{R}^n \} \otimes_\mathbb{R} \Omega^*.
    \end{equation}
    We have $\Omega^*(\mathbb{R}^n) = \oplus^n_{q=0} \Omega^q(\mathbb{R}^n)$,
    where $\Omega^q(\mathbb{R}^n)$ consists of the $C^\infty$ $q$-forms on $\mathbb{R}^n$. We define
    \begin{equation}
        d : \Omega^q(\mathbb{R}^n) \rightarrow \Omega^{q+1}(\mathbb{R}^n),
    \end{equation}
    the exterior differentiation, by
    \setlist{nolistsep}
    \begin{enumerate}[noitemsep]
        \item if $f \in \Omega^0(\mathbb{R}^n)$, then $df = \sum \partial f / \partial x_i \, dx_i$,
        \item if $\omega = \sum f_I dx_I$, then $d\omega = \sum df_I dx_I$, where $dx_I = dx_i dx_j \dots $.
    \end{enumerate}
    The wedge product is defined by
    \begin{equation}
        \tau \wedge \omega = \sum \tau_I \omega_J \, dx_I dx_J = (-1)^{\mathrm{deg }\tau\,\mathrm{deg }\omega} \omega \wedge \tau.
    \end{equation}
\end{definition}
\begin{proposition}
    $d$ is an antiderivation,
    \begin{equation}
        d(\tau \wedge \omega) = (d\tau) \wedge \omega + (-1)^{\mathrm{deg} \tau} \tau \wedge d\omega.
    \end{equation}
\end{proposition}
\begin{proposition}
    $d^2 = 0$.
\end{proposition}
\begin{definition}
    The $q$-th de Rham cohomology of $\mathbb{R}^n$ is the vector space
    \begin{equation}
        H^q(\mathbb{R}^n) = \{\textrm{closed $q$-forms}\}/\{\textrm{exact $q$-forms}\},
    \end{equation}
    where closed means in the kernel of $d$ and exact means in the image of $d$.
    We denote by $[\omega]$ the cohomology class of $\omega$.
\end{definition}
\begin{definition}
    A differential complex is a direct sum of Vector spaces $C = \oplus_{q\in\mathbb{Z}} C^q$ if there are homomorpisms
    \begin{center}
    \begin{tikzcd}[cells={nodes={minimum height=2em}}]
       \dots \arrow[r] & C^{q-1} \arrow[r, "d"] & C^i \arrow[r, "d"] & C^{q+1} \arrow[r] & \dots
    \end{tikzcd}
    \end{center}
    with $d^2 = 0$. The cohomology of $C$ is given by $H(C) = \oplus_{q\in\mathbb{Z}} H^q(C)$, with
    \begin{equation}
        H^q(C) = (\textrm{ker } d \cap C^q)/(\textrm{im } d \cap C^q).
    \end{equation}
    A map $f: A \rightarrow B$ between two differential complexes is a chain map it it commutes with the differential operators of A and B, $fd_A=d_Bf$.
    A sequence of vector spaces
    \begin{center}
    \begin{tikzcd}[cells={nodes={minimum height=2em}}]
       \dots \arrow[r] & V_{q-1} \arrow[r, "f_{i-1}"] & V_i \arrow[r, "f_i"] & V_{q+1} \arrow[r] & \dots
    \end{tikzcd}
    \end{center}
    is said to be exact if the image of $f_{i-1}$ is the kernel of $f_{i}$.
    An exact sequence of the form
    \begin{center}
    \begin{tikzcd}[cells={nodes={minimum height=2em}}]
       0 \arrow[r] & A \arrow[r, "f"] & B \arrow[r, "g"] & C \arrow[r] & 0
    \end{tikzcd}
    \end{center}
    is called a short exact sequence. Note that $f$ is injective and $g$ surjective.
    If $f$, $g$ are chain maps, there is a long exact sequence of cohomology groups
    \begin{center}
    \begin{tikzcd}[cells={nodes={minimum height=2em}}]
        H^{q+1}(A) \arrow[r, "f^*"] & ... & \\
        H^q(A) \arrow[r, "f^*"] & H^q(B) \arrow[r, "g^*"] & H^q(C) \ar["d^*", out=90, in=270]{ull}
    \end{tikzcd}
    \end{center}
    $f^*$, $g^*$ are the naturally induced maps and $d^*[c]$ is obtained through the commutative diagram
    \begin{center}
    \begin{tikzcd}[cells={nodes={minimum height=2em}}]
        0 \arrow[r] & A^{q+1} \arrow[r,"f"] & B^{q+1} \arrow[r,"g"] & C^{q+1} \arrow[r] & 0\\
        0 \arrow[r] & A^q \arrow[r,"f"] \arrow[u, "d"] & B^q \arrow[r,"g"] \arrow[u, "d"] & C^q \arrow[r] \arrow[u, "d"] & 0
    \end{tikzcd}
    \end{center}
    Since $g$ is surjective there is $b\in B^q$ with $g(b)=c$. Because $g(db)=d(gb)=dc=0$, there is $a\in A^{q+1}$ with $db=f(a)$.
Then $d^*[c] := [a]$. $a$ is closed because $f$ is injective. To see that the sequence is exact, note that if $b$ is closed then $f(a) = 0$, and due to injectivity $a = 0$.
    On the other hand, $f(a)$ is exact and therefore $[f(a)] = 0$.
\end{definition}



\end{document}