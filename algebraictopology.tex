\documentclass[twocolumn]{article}
\usepackage[a4paper, total={7.5in, 10.7in}]{geometry}
\usepackage{amsthm}
\usepackage{amssymb}
\usepackage{amsmath}
\usepackage{enumitem}
\usepackage{tikz-cd}

\theoremstyle{definition}
\newtheorem{definition}{Definition}[section]
\newtheorem{theorem}[definition]{Theorem}
\newtheorem{lemma}[definition]{Lemma}
\newtheorem{proposition}[definition]{Proposition}
\newtheorem{corollary}[definition]{Corollary}
\theoremstyle{remark}
\newtheorem*{remark}{Remark}

\begin{document}

\section{Algebraic Topology}
\begin{definition}[de Rham complex]
    $\Omega^*$ is the algebra generated over $\mathbb{R}$ by $dx_1, \dots, dx_n$ subject to
    \setlist{nolistsep}
    \begin{enumerate}[noitemsep]
        \item $(dx_i)^2 = 0$,
        \item $dx_i dx_j = -dx_j dx_i, i \neq j$.
    \end{enumerate}
    The $C^\infty$ differential forms on $\mathbb{R}$ are elements of
    \begin{equation}
        \Omega^*(\mathbb{R}^n) = \{ C^\infty \textrm{ functions on } \mathbb{R}^n \} \otimes_\mathbb{R} \Omega^*.
    \end{equation}
    We have $\Omega^*(\mathbb{R}^n) = \oplus^n_{q=0} \Omega^q(\mathbb{R}^n)$,
    where $\Omega^q(\mathbb{R}^n)$ consists of the $C^\infty$ $q$-forms on $\mathbb{R}^n$. We define
    \begin{equation}
        d : \Omega^q(\mathbb{R}^n) \rightarrow \Omega^{q+1}(\mathbb{R}^n),
    \end{equation}
    the exterior differentiation, by
    \setlist{nolistsep}
    \begin{enumerate}[noitemsep]
        \item if $f \in \Omega^0(\mathbb{R}^n)$, then $df = \sum \partial f / \partial x_i \, dx_i$,
        \item if $\omega = \sum f_I dx_I$, then $d\omega = \sum df_I dx_I$, where $dx_I = dx_i dx_j \dots $.
    \end{enumerate}
    The wedge product is defined by
    \begin{equation}
        \tau \wedge \omega = \sum \tau_I \omega_J \, dx_I dx_J = (-1)^{\mathrm{deg }\tau\,\mathrm{deg }\omega} \omega \wedge \tau.
    \end{equation}
\end{definition}
\begin{proposition}
    $d$ is an antiderivation,
    \begin{equation}
        d(\tau \wedge \omega) = (d\tau) \wedge \omega + (-1)^{\mathrm{deg} \tau} \tau \wedge d\omega.
    \end{equation}
\end{proposition}
\begin{proposition}
    $d^2 = 0$.
\end{proposition}
\begin{definition}
    The $q$-th de Rham cohomology of $\mathbb{R}^n$ is the vector space
    \begin{equation}
        H^q(\mathbb{R}^n) = \{\textrm{closed $q$-forms}\}/\{\textrm{exact $q$-forms}\},
    \end{equation}
    where closed means in the kernel of $d$ and exact means in the image of $d$.
    We denote by $[\omega]$ the cohomology class of $\omega$.
\end{definition}
\begin{remark}
    Only the constant functions are relevant for $\mathbb{R}^n$,
    \begin{equation}
        H^*(\mathbb{R}^n) =
        \begin{cases}
            R & \textrm{in dimension 0,}\\
            0 & \textrm{otherwise.}
        \end{cases}
    \end{equation}
\end{remark}
\begin{definition}
    A differential complex is a direct sum of Vector spaces $C = \oplus_{q\in\mathbb{Z}} C^q$ if there are homomorpisms
    \begin{center}
    \begin{tikzcd}[cells={nodes={minimum height=2em}}]
       \dots \arrow[r] & C^{q-1} \arrow[r, "d"] & C^i \arrow[r, "d"] & C^{q+1} \arrow[r] & \dots
    \end{tikzcd}
    \end{center}
    with $d^2 = 0$. The cohomology of $C$ is given by $H(C) = \oplus_{q\in\mathbb{Z}} H^q(C)$, with
    \begin{equation}
        H^q(C) = (\textrm{ker } d \cap C^q)/(\textrm{im } d \cap C^q).
    \end{equation}
    A map $f: A \rightarrow B$ between two differential complexes is a chain map it it commutes with the differential operators of A and B, $fd_A=d_Bf$.
    A sequence of vector spaces
    \begin{center}
    \begin{tikzcd}[cells={nodes={minimum height=2em}}]
       \dots \arrow[r] & V_{q-1} \arrow[r, "f_{i-1}"] & V_i \arrow[r, "f_i"] & V_{q+1} \arrow[r] & \dots
    \end{tikzcd}
    \end{center}
    is said to be exact if the image of $f_{i-1}$ is the kernel of $f_{i}$.
    An exact sequence of the form
    \begin{center}
    \begin{tikzcd}[cells={nodes={minimum height=2em}}]
       0 \arrow[r] & A \arrow[r, "f"] & B \arrow[r, "g"] & C \arrow[r] & 0
    \end{tikzcd}
    \end{center}
    is called a short exact sequence. Note that $f$ is injective and $g$ surjective.
    If $f$, $g$ are chain maps, there is a long exact sequence of cohomology groups
    \begin{center}
    \begin{tikzcd}[cells={nodes={minimum height=2em}}]
        H^{q+1}(A) \arrow[r, "f^*"] & ... & \\
        H^q(A) \arrow[r, "f^*"] & H^q(B) \arrow[r, "g^*"] & H^q(C) \ar["d^*", out=90, in=270]{ull}
    \end{tikzcd}
    \end{center}
    $f^*$, $g^*$ are the naturally induced maps and $d^*[c]$ is obtained through the commutative diagram
    \begin{center}
    \begin{tikzcd}[cells={nodes={minimum height=2em}}]
        0 \arrow[r] & A^{q+1} \arrow[r,"f"] & B^{q+1} \arrow[r,"g"] & C^{q+1} \arrow[r] & 0\\
        0 \arrow[r] & A^q \arrow[r,"f"] \arrow[u, "d"] & B^q \arrow[r,"g"] \arrow[u, "d"] & C^q \arrow[r] \arrow[u, "d"] & 0
    \end{tikzcd}
    \end{center}
    Since $g$ is surjective there is $b\in B^q$ with $g(b)=c$. Because $g(db)=d(gb)=dc=0$, there is $a\in A^{q+1}$ with $db=f(a)$.
    Then $d^*[c] := [a]$. $a$ is closed because $f$ is injective. To see that the sequence is exact, note that if $b$ is closed then $f(a) = 0$, and due to injectivity $a = 0$.
    On the other hand, $f(a)$ is exact and therefore $[f(a)] = 0$.
\end{definition}
\begin{definition}
    $\Omega_c^*(\mathbb{R}^n)$ is the de Rham complex for functions of compact support,
    $H_c^*(\mathbb{R})$ is its cohomology.
\end{definition}
\begin{remark}
    Only the $n$-forms whose integrals are different from zero are relevant,
    since if it was zero then its antiderivative can have compact support,
    \begin{equation}
        H_c^*(\mathbb{R}^n) =
        \begin{cases}
            R & \textrm{in dimension n,}\\
            0 & \textrm{otherwise.}
        \end{cases}
    \end{equation}
\end{remark}
\begin{definition}
    $f: \mathbb{R}^m \rightarrow \mathbb{R}^n$ induces a pullback on functions
    \begin{equation}
        f^*(g) = g \circ f.
    \end{equation}
    On forms the pullback is defined as
    \begin{equation}
        f^*(\sum g_I dy_{i_1} \dots dy_{i_q}) = \sum(g_I \circ f) df_{i_1} \dots df_{i_q},
    \end{equation}
    with $f_i = y_i \circ f$ the i-th component of $f$, $y_i$ the standard coordinates.
\end{definition}
\begin{proposition}
    $f^*$ commutes with $d$. This shows that the cohomology is a diffeomorphism invariant.
\end{proposition}
\begin{definition}
    Let $M = U \cup V$ with $U, V$ open. Then we have the inclusions
    \begin{equation}
        M \leftarrow U \sqcup V \leftarrow U \cap V
    \end{equation}
    where $\sqcup$ is the disjoint union(each element has a label indicating wether it's from $U$ or $V$).
    Using the inclusions as pullbacks we get
    \begin{equation}
        \Omega^*(M) \rightarrow \Omega^*(U) \oplus \Omega^*(V) \rightarrow \Omega^*(U \cap V).
    \end{equation}
    The Mayer-Vietoris sequence is given by
    \begin{equation}
        0 \rightarrow \Omega^*(M) \rightarrow \Omega^*(U) \oplus \Omega^*(V) \rightarrow \Omega^*(U \cap V) \rightarrow 0,
    \end{equation}
    with
    \begin{equation}
        \Omega^*(U) \oplus \Omega^*(V) \rightarrow \Omega^*(U \cap V); (\omega, \tau) \mapsto \tau - \omega.
    \end{equation}
\end{definition}
\begin{proposition}
    The Mayer-Vietoris sequence is exact. This is achieved through partitions of unity $\rho$,
    \begin{equation}
        (\rho_U \omega) - (-\rho_V \omega) = \omega.
    \end{equation}
\end{proposition}
\begin{definition}
    The Mayer-Vietoris sequence induces a long exact sequence with the same name:
    \begin{center}
    \begin{tikzcd}
        H^{q+1}(M) \arrow[r] & H^{q+1}(U) \oplus H^{q+1}(V) \arrow[r] & H^{q+1}(U \cap V) \\
        H^q(M) \arrow[r] & H^q(U) \oplus H^q(V) \arrow[r] & H^q(U \cap V) \ar["d^*"]{ull}
    \end{tikzcd}
    \end{center}
    Explicitly
    \begin{equation}
        d^*[\omega] = 
        \begin{cases}
            [-d(\rho_V\omega)] & \textrm{on } U,\\
            [d(\rho_U\omega)] & \textrm{on } V.
        \end{cases}
    \end{equation}
\end{definition}
\begin{definition}
    If $j: U \rightarrow M$ is the inclusion of $U$ in $M$, then let $j_*: \Omega^*_c(U) \rightarrow \Omega^*_c(M)$ the map which extends a form to $M$ by zero.
    Because pullbacks of compact forms are in general not compact, we instead use the inclusions

    \begin{center}
    \begin{tikzcd}
       \Omega^*_c(M) & \arrow[l, "\textrm{sum}"] \Omega^*_c(U) \oplus \Omega^*_c(V) & \arrow[l, "\delta"] \Omega^*_c(U \cap V)
    \end{tikzcd}
    \end{center}
    \begin{equation}
        \delta: \omega \mapsto (-j_*\omega, j_*\omega).
    \end{equation}
    We then get the Mayer-Vietoris sequence
    \begin{equation}
        0 \leftarrow \Omega^*_c(M) \leftarrow \Omega^*_c(U) \oplus \Omega^*_c(V) \leftarrow \Omega^*_c(U \cap V) \leftarrow 0,
    \end{equation}
    which induces
    \begin{center}
    \begin{tikzcd}
        H^{q+1}_c(M) & \arrow[l] H^{q+1}_c(U) \oplus H^{q+1}_c(V) & \arrow[l] H^{q+1}_c(U \cap V)\\
        H^q_c(M) \ar["d^*"]{urr} & \arrow[l] H^q_c(U) \oplus H^q_c(V) & \arrow[l] H^q_c(U \cap V)
    \end{tikzcd}
    \end{center}
    and we now instead get
    \begin{equation}
        d^*[\omega] = 
        \begin{cases}
            [d(\rho_U\omega)] & \textrm{on } U,\\
            [d(\rho_V\omega)] & \textrm{on } V.
        \end{cases}
    \end{equation}
\end{definition}
\begin{proposition}
    The Mayer-Vietoris sequence of forms with compact support is exact.
\end{proposition}
\begin{proposition}
    A manifold of dimension $n$ is orientable iff it has a global nowhere vanishing $n$-form.
\end{proposition}
\begin{definition}
    Let $\mathbb{H}^n = \{x_n \geq 0 \} \subset \mathbb{R}^n$ with standard orientation $dx_1\dots dx_n$.
    The induced orientation of $\partial \mathbb{H}^n = \{x_n=0\}$ is given by the equivalence class of $(-1)^n dx_1 \dots dx_{n-1}$.
    For an orientation-preserving diffeomorphism $\phi$ we define for manifolds
    \begin{equation}
        [\partial M] = \phi^*[\partial \mathbb{H}^n].
    \end{equation}
\end{definition}
\begin{remark}
    This definition is due to $\omega|_{\partial M} := i_{\hat{n}} \omega$ for the normal $\hat{n}$.
\end{remark}
\begin{theorem}[Stokes']
    Let $\omega$ be an $(n-1)$-form with compact support on an oriented manifold $M$, then
    \begin{equation}
        \int_M d\omega = \int_{\partial M} \omega.
    \end{equation}
\end{theorem}

\begin{definition}
    Let $\pi: \mathbb{R}^n \times \mathbb{R}^1 \rightarrow \mathbb{R}^n;\, \pi(x, t) = x$ and
    $s: \mathbb{R}^n \rightarrow \mathbb{R}^n \times \mathbb{R}^1;\, s(x) = (x, 0)$.
    Trivially $\pi \circ s = 1$, $s^* \circ \pi^* = 1$, but $s \circ \pi \neq 1$.
    $K$ is called a homotopy operator if
    \begin{equation}
        1 - \pi^* \circ s^* = \pm(dK \pm Kd).
    \end{equation}
    $dK \pm Kd$ maps closed forms to exact forms, therefore induces zero in cohomology.
    If $K$ exists, $\pi^* \circ s^*$ is said to be chain homotopic to the identity.
    We define $K: \Omega^q(\mathbb{R}^n \times \mathbb{R}) \rightarrow \Omega^{q-1}(\mathbb{R}^n \times \mathbb{R})$ by
    \begin{equation}
        (\pi^* \phi)f(x, t) \mapsto 0,\quad
        (\pi^* \phi)f(x, t) dt \mapsto (\pi^* \phi) \int^t_0 dt\, f
    \end{equation}
    with $\phi$ a form on $\mathbb{R}^n$.
\end{definition}
\begin{proposition}
    $K$ is a homotopy operator.
    The maps $\pi^*$, $s^*$ on $H^*(\mathbb{R}^n \times \mathbb{R}) \leftrightarrow H^*(\mathbb{R}^n)$ are isomorphisms.
\end{proposition}
\begin{corollary}[Poincaré Lemma]
    \begin{equation}
        H^*(\mathbb{R}^n) =
        \begin{cases}
            R & \textrm{in dimension 0,}\\
            0 & \textrm{otherwise.}
        \end{cases}
    \end{equation}
    More generally $H^*(M \times \mathbb{R}^1) \simeq H^*(M)$.
\end{corollary}
\begin{corollary}[Homotopy Axiom for de Rham Cohomology]
    Homotopic maps induce the same map in cohomology.
\end{corollary}




\end{document}