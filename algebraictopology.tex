\documentclass[twocolumn]{article}
\usepackage[a4paper, total={7.5in, 10.7in}]{geometry}
\usepackage{amsthm}
\usepackage{amssymb}
\usepackage{amsmath}
\usepackage{enumitem}
\usepackage{tikz-cd}
\tikzcdset{scale cd/.style={every label/.append style={scale=#1},
    cells={nodes={scale=#1}}}}

\theoremstyle{definition}
\newtheorem{definition}{Definition}[section]
\newtheorem{theorem}[definition]{Theorem}
\newtheorem{lemma}[definition]{Lemma}
\newtheorem{proposition}[definition]{Proposition}
\newtheorem{corollary}[definition]{Corollary}
\theoremstyle{remark}
\newtheorem*{remark}{Remark}

\begin{document}

\section{Algebraic Topology}
\begin{definition}[de Rham complex]
    $\Omega^*$ is the algebra generated over $\mathbb{R}$ by $dx_1, \dots, dx_n$ subject to
    \setlist{nolistsep}
    \begin{enumerate}[noitemsep]
        \item $(dx_i)^2 = 0$,
        \item $dx_i dx_j = -dx_j dx_i, i \neq j$.
    \end{enumerate}
    The $C^\infty$ differential forms on $\mathbb{R}$ are elements of
    \begin{equation}
        \Omega^*(\mathbb{R}^n) = \{ C^\infty \textrm{ functions on } \mathbb{R}^n \} \otimes_\mathbb{R} \Omega^*.
    \end{equation}
    We have $\Omega^*(\mathbb{R}^n) = \oplus^n_{q=0} \Omega^q(\mathbb{R}^n)$,
    where $\Omega^q(\mathbb{R}^n)$ consists of the $C^\infty$ $q$-forms on $\mathbb{R}^n$. We define
    \begin{equation}
        d : \Omega^q(\mathbb{R}^n) \rightarrow \Omega^{q+1}(\mathbb{R}^n),
    \end{equation}
    the exterior differentiation, by
    \setlist{nolistsep}
    \begin{enumerate}[noitemsep]
        \item if $f \in \Omega^0(\mathbb{R}^n)$, then $df = \sum \partial f / \partial x_i \, dx_i$,
        \item if $\omega = \sum f_I dx_I$, then $d\omega = \sum df_I dx_I$, where $dx_I = dx_i dx_j \dots $.
    \end{enumerate}
    The wedge product is defined by
    \begin{equation}
        \tau \wedge \omega = \sum \tau_I \omega_J \, dx_I dx_J = (-1)^{\mathrm{deg }\tau\,\mathrm{deg }\omega} \omega \wedge \tau.
    \end{equation}
\end{definition}
\begin{proposition}
    $d$ is an antiderivation,
    \begin{equation}
        d(\tau \wedge \omega) = (d\tau) \wedge \omega + (-1)^{\mathrm{deg} \tau} \tau \wedge d\omega.
    \end{equation}
\end{proposition}
\begin{proposition}
    $d^2 = 0$.
\end{proposition}
\begin{definition}
    The $q$-th de Rham cohomology of $\mathbb{R}^n$ is the vector space
    \begin{equation}
        H^q(\mathbb{R}^n) = \{\textrm{closed $q$-forms}\}/\{\textrm{exact $q$-forms}\},
    \end{equation}
    where closed means in the kernel of $d$ and exact means in the image of $d$.
    We denote by $[\omega]$ the cohomology class of $\omega$.
\end{definition}
\begin{remark}
    Only the constant functions are relevant for $\mathbb{R}^n$,
    \begin{equation}
        H^*(\mathbb{R}^n) =
        \begin{cases}
            R & \textrm{in dimension 0,}\\
            0 & \textrm{otherwise.}
        \end{cases}
    \end{equation}
\end{remark}
\begin{definition}
    A differential complex is a direct sum of Vector spaces $C = \oplus_{q\in\mathbb{Z}} C^q$ if there are homomorpisms
    \begin{center}
    \begin{tikzcd}[cells={nodes={minimum height=2em}}]
       \dots \arrow[r] & C^{q-1} \arrow[r, "d"] & C^i \arrow[r, "d"] & C^{q+1} \arrow[r] & \dots
    \end{tikzcd}
    \end{center}
    with $d^2 = 0$. The cohomology of $C$ is given by $H(C) = \oplus_{q\in\mathbb{Z}} H^q(C)$, with
    \begin{equation}
        H^q(C) = (\textrm{ker } d \cap C^q)/(\textrm{im } d \cap C^q).
    \end{equation}
    A map $f: A \rightarrow B$ between two differential complexes is a chain map it it commutes with the differential operators of A and B, $fd_A=d_Bf$.
    A sequence of vector spaces
    \begin{center}
    \begin{tikzcd}[cells={nodes={minimum height=2em}}]
       \dots \arrow[r] & V_{q-1} \arrow[r, "f_{i-1}"] & V_i \arrow[r, "f_i"] & V_{q+1} \arrow[r] & \dots
    \end{tikzcd}
    \end{center}
    is said to be exact if the image of $f_{i-1}$ is the kernel of $f_{i}$.
    An exact sequence of the form
    \begin{center}
    \begin{tikzcd}[cells={nodes={minimum height=2em}}]
       0 \arrow[r] & A \arrow[r, "f"] & B \arrow[r, "g"] & C \arrow[r] & 0
    \end{tikzcd}
    \end{center}
    is called a short exact sequence. Note that $f$ is injective and $g$ surjective.
    If $f$, $g$ are chain maps, there is a long exact sequence of cohomology groups
    \begin{center}
    \begin{tikzcd}[cells={nodes={minimum height=2em}}]
        H^{q+1}(A) \arrow[r, "f^*"] & ... & \\
        H^q(A) \arrow[r, "f^*"] & H^q(B) \arrow[r, "g^*"] & H^q(C) \ar["d^*", out=90, in=270]{ull}
    \end{tikzcd}
    \end{center}
    $f^*$, $g^*$ are the naturally induced maps and $d^*[c]$ is obtained through the commutative diagram
    \begin{center}
    \begin{tikzcd}[cells={nodes={minimum height=2em}}]
        0 \arrow[r] & A^{q+1} \arrow[r,"f"] & B^{q+1} \arrow[r,"g"] & C^{q+1} \arrow[r] & 0\\
        0 \arrow[r] & A^q \arrow[r,"f"] \arrow[u, "d"] & B^q \arrow[r,"g"] \arrow[u, "d"] & C^q \arrow[r] \arrow[u, "d"] & 0
    \end{tikzcd}
    \end{center}
    Since $g$ is surjective there is $b\in B^q$ with $g(b)=c$. Because $g(db)=d(gb)=dc=0$, there is $a\in A^{q+1}$ with $db=f(a)$.
    Then $d^*[c] := [a]$. $a$ is closed because $f$ is injective. To see that the sequence is exact, note that if $b$ is closed then $f(a) = 0$, and due to injectivity $a = 0$.
    On the other hand, $f(a)$ is exact and therefore $[f(a)] = 0$.
\end{definition}
\begin{definition}
    $\Omega_c^*(\mathbb{R}^n)$ is the de Rham complex for functions of compact support,
    $H_c^*(\mathbb{R})$ is its cohomology.
\end{definition}
\begin{remark}
    Only the $n$-forms whose integrals are different from zero are relevant,
    since if it was zero then its antiderivative can have compact support,
    \begin{equation}
        H_c^*(\mathbb{R}^n) =
        \begin{cases}
            R & \textrm{in dimension n,}\\
            0 & \textrm{otherwise.}
        \end{cases}
    \end{equation}
\end{remark}
\begin{definition}
    $f: \mathbb{R}^m \rightarrow \mathbb{R}^n$ induces a pullback on functions
    \begin{equation}
        f^*(g) = g \circ f.
    \end{equation}
    On forms the pullback is defined as
    \begin{equation}
        f^*(\sum g_I dy_{i_1} \dots dy_{i_q}) = \sum(g_I \circ f) df_{i_1} \dots df_{i_q},
    \end{equation}
    with $f_i = y_i \circ f$ the i-th component of $f$, $y_i$ the standard coordinates.
\end{definition}
\begin{proposition}
    $f^*$ commutes with $d$. This shows that the cohomology is a diffeomorphism invariant.
\end{proposition}
\begin{definition}
    Let $M = U \cup V$ with $U, V$ open. Then we have the inclusions
    \begin{equation}
        M \leftarrow U \sqcup V \leftarrow U \cap V
    \end{equation}
    where $\sqcup$ is the disjoint union(each element has a label indicating wether it's from $U$ or $V$).
    Using the inclusions as pullbacks we get
    \begin{equation}
        \Omega^*(M) \rightarrow \Omega^*(U) \oplus \Omega^*(V) \rightarrow \Omega^*(U \cap V).
    \end{equation}
    The Mayer-Vietoris sequence is given by
    \begin{equation}
        0 \rightarrow \Omega^*(M) \rightarrow \Omega^*(U) \oplus \Omega^*(V) \rightarrow \Omega^*(U \cap V) \rightarrow 0,
    \end{equation}
    with
    \begin{equation}
        \Omega^*(U) \oplus \Omega^*(V) \rightarrow \Omega^*(U \cap V); (\omega, \tau) \mapsto \tau - \omega.
    \end{equation}
\end{definition}
\begin{proposition}
    The Mayer-Vietoris sequence is exact. This is achieved through partitions of unity $\rho$,
    \begin{equation}
        (\rho_U \omega) - (-\rho_V \omega) = \omega.
    \end{equation}
\end{proposition}
\begin{definition}
    The Mayer-Vietoris sequence induces a long exact sequence with the same name:
    \begin{center}
    \begin{tikzcd}
        H^{q+1}(M) \arrow[r] & H^{q+1}(U) \oplus H^{q+1}(V) \arrow[r] & H^{q+1}(U \cap V) \\
        H^q(M) \arrow[r] & H^q(U) \oplus H^q(V) \arrow[r] & H^q(U \cap V) \ar["d^*"]{ull}
    \end{tikzcd}
    \end{center}
    Explicitly
    \begin{equation}
        d^*[\omega] = 
        \begin{cases}
            [-d(\rho_V\omega)] & \textrm{on } U,\\
            [d(\rho_U\omega)] & \textrm{on } V.
        \end{cases}
    \end{equation}
\end{definition}
\begin{definition}
    If $j: U \rightarrow M$ is the inclusion of $U$ in $M$, then let $j_*: \Omega^*_c(U) \rightarrow \Omega^*_c(M)$ the map which extends a form to $M$ by zero.
    Because pullbacks of compact forms are in general not compact, we instead use the inclusions
    \begin{center}
    \begin{tikzcd}
       \Omega^*_c(M) & \arrow[l, "\textrm{sum}"] \Omega^*_c(U) \oplus \Omega^*_c(V) & \arrow[l, "\delta"] \Omega^*_c(U \cap V)
    \end{tikzcd}
    \end{center}
    \begin{equation}
        \delta: \omega \mapsto (-j_*\omega, j_*\omega).
    \end{equation}
    We then get the Mayer-Vietoris sequence
    \begin{equation}
        0 \leftarrow \Omega^*_c(M) \leftarrow \Omega^*_c(U) \oplus \Omega^*_c(V) \leftarrow \Omega^*_c(U \cap V) \leftarrow 0,
    \end{equation}
    which induces
    \begin{center}
    \begin{tikzcd}
        H^{q+1}_c(M) & \arrow[l] H^{q+1}_c(U) \oplus H^{q+1}_c(V) & \arrow[l] H^{q+1}_c(U \cap V)\\
        H^q_c(M) \ar["d^*"]{urr} & \arrow[l] H^q_c(U) \oplus H^q_c(V) & \arrow[l] H^q_c(U \cap V)
    \end{tikzcd}
    \end{center}
    and we now instead get
    \begin{equation}
        d^*[\omega] = 
        \begin{cases}
            [-d(\rho_U\omega)] & \textrm{on } U,\\
            [d(\rho_V\omega)] & \textrm{on } V.
        \end{cases}
    \end{equation}
\end{definition}
\begin{proposition}
    The Mayer-Vietoris sequence of forms with compact support is exact.
\end{proposition}
\begin{proposition}
    A manifold of dimension $n$ is orientable iff it has a global nowhere vanishing $n$-form.
\end{proposition}
\begin{definition}
    Let $\mathbb{H}^n = \{x_n \geq 0 \} \subset \mathbb{R}^n$ with standard orientation $dx_1\dots dx_n$.
    The induced orientation of $\partial \mathbb{H}^n = \{x_n=0\}$ is given by the equivalence class of $(-1)^n dx_1 \dots dx_{n-1}$.
    For an orientation-preserving diffeomorphism $\phi$ we define for manifolds
    \begin{equation}
        [\partial M] = \phi^*[\partial \mathbb{H}^n].
    \end{equation}
\end{definition}
\begin{remark}
    This definition is due to $\omega|_{\partial M} := i_{\hat{n}} \omega$ for the normal $\hat{n}$.
\end{remark}
\begin{theorem}[Stokes']
    Let $\omega$ be an $(n-1)$-form with compact support on an oriented manifold $M$, then
    \begin{equation}
        \int_M d\omega = \int_{\partial M} \omega.
    \end{equation}
\end{theorem}

\begin{definition}
    Let $\pi: \mathbb{R}^n \times \mathbb{R}^1 \rightarrow \mathbb{R}^n;\, \pi(x, t) = x$ and
    $s: \mathbb{R}^n \rightarrow \mathbb{R}^n \times \mathbb{R}^1;\, s(x) = (x, 0)$.
    Trivially $\pi \circ s = 1$, $s^* \circ \pi^* = 1$, but $s \circ \pi \neq 1$.
    $K$ is called a homotopy operator if
    \begin{equation}
        1 - \pi^* \circ s^* = \pm(dK \pm Kd).
    \end{equation}
    $dK \pm Kd$ maps closed forms to exact forms, therefore induces zero in cohomology.
    If $K$ exists, $\pi^* \circ s^*$ is said to be chain homotopic to the identity.
    We define $K: \Omega^q(\mathbb{R}^n \times \mathbb{R}) \rightarrow \Omega^{q-1}(\mathbb{R}^n \times \mathbb{R})$ by
    \begin{equation}
        (\pi^* \phi)f(x, t) \mapsto 0,\quad
        (\pi^* \phi)f(x, t) dt \mapsto (\pi^* \phi) \int^t_0 dt\, f
    \end{equation}
    with $\phi$ a form on $\mathbb{R}^n$.
\end{definition}
\begin{proposition}
    $K$ is a homotopy operator.
    The maps $\pi^*$, $s^*$ on $H^*(\mathbb{R}^n \times \mathbb{R}) \leftrightarrow H^*(\mathbb{R}^n)$ are isomorphisms.
\end{proposition}
\begin{corollary}[Poincaré Lemma]
    \begin{equation}
        H^*(\mathbb{R}^n) =
        \begin{cases}
            R & \textrm{in dimension 0,}\\
            0 & \textrm{otherwise.}
        \end{cases}
    \end{equation}
    More generally $H^*(M \times \mathbb{R}^1) \simeq H^*(M)$.
\end{corollary}
\begin{corollary}[Homotopy Axiom for de Rham Cohomology]
    \label{homotopyAxiom}
    Homotopic maps induce the same map in cohomology.
\end{corollary}
\begin{definition}
    Two Manifolds $M$, $N$ have the same homotopy type if there are $C^\infty$ maps
    $f: M \rightarrow N$, $g: N \rightarrow M$ such that $g \circ f$, $f \circ g$ are $C^\infty$
    homotopic to the identity on $M$, $N$ respectively. A manifold is called contractible if it has the homotopy type of a point.
\end{definition}
\begin{corollary}
    Manifolds with the same homotopy type have the same de Rham cohomology.
\end{corollary}
\begin{definition}
    $r: M \rightarrow A$ is called a retraction of $M$ onto $A$ if $r \circ i: A \rightarrow A$ is the identity,
    with $i: A \subset M$ the inclusion. If $i \circ r: M \rightarrow M$ is homotopic to the identitiy on M,
    then r is called a deformation retraction of $M$ onto $A$, and it follows that $M$, $A$ have the same homotopy type.
\end{definition}
\begin{corollary}
    If $A$ is a deformation retract of $M$, then $A$, $M$ have the same de Rham cohomology.
\end{corollary}
\begin{corollary}
    \begin{equation}
        H^*(S^n) =
        \begin{cases}
            R & \textrm{in dimension 0, n}\\
            0 & \textrm{otherwise.}
        \end{cases}
    \end{equation}
\end{corollary}
\begin{definition}
    Let $\pi: M \times \mathbb{R}^1 \rightarrow M$ be the projection.
    We define the push-forward for compact forms $\pi_*: \Omega_c^*(M \times \mathbb{R}^1) \rightarrow \Omega^{*-1}_c(M)$, called integration along the fiber by
    \begin{equation}
        \pi^*\phi f(x, t) \mapsto 0, \quad \pi^*\phi f(x, t) dt \mapsto \phi \int^\infty_{-\infty} dt\,f
    \end{equation}
    where $\phi$ is a form on $M$. Let $e = e(t) dt$ be a compactly supported 1-form on $\mathbb{R}^1$ with total integral 1 and define
    \begin{equation}
        e_*: \Omega^*_c(M) \rightarrow \Omega^{*+1}_c(M\times \mathbb{R}^1); \quad \phi \rightarrow (\pi^*\phi) \wedge e.
    \end{equation}
    The homotopy operator $K: \Omega_c^*(M \times \mathbb{R}^1) \rightarrow\Omega_c^{*-1}(M \times \mathbb{R}^1)$ is defined by
    \begin{equation}
        \pi^*\phi f \mapsto 0, \quad \pi^*\phi f dt \mapsto \pi^*\phi \int^t_{-\infty} f - \pi^*\phi \int^t_{-\infty} e \int^\infty_{-\infty}f.
    \end{equation}
\end{definition}
\begin{proposition}
    $d$ commutes with both $\pi_*$ and $e_*$.
\end{proposition}
\begin{proposition}
    $1 - e_* \pi_* = (-1)^{q-1}(dK - Kd)$ on $\Omega^q_c(M \times \mathbb{R}^1)$.
    The maps $H_c^*(M\times \mathbb{R}^1) \leftrightarrow H_d^{*-1}(M)$ are isomorphisms.
\end{proposition}
\begin{corollary}[Poincaré Lemma for Compact Supports]
    \begin{equation}
        H^*_c(\mathbb{R}^n) =
        \begin{cases}
            R & \textrm{in dimension n}\\
            0 & \textrm{otherwise.}
        \end{cases}
    \end{equation}
    The generator is an $n$-form of compact support with integral 1.
\end{corollary}
\begin{definition}
    A map is proper if the inverse image of every compact set is compact.
\end{definition}
\begin{theorem}[Sard's]
    The set of critical values of a smooth map $f: M \rightarrow N$ has measure zero.
\end{theorem}
\begin{definition}
    Let $f: \mathbb{R}^n \rightarrow \mathbb{R}^n$ be proper. Let $\alpha$ be a generator
    of $H^n_c(\mathbb{R}^n)$(integrates to 1), then
    \begin{equation}
        \textrm{deg }f = \int_{\mathbb{R}^n} f^*\alpha
    \end{equation}
    is the degree of $f$.
\end{definition}
\begin{proposition}
    Let $f: \mathbb{R}^n \rightarrow \mathbb{R}^n$ be proper. If $f$ is not surjective, then it has degree 0.
\end{proposition}
\begin{theorem}
    Let $f: \mathbb{R}^n \rightarrow \mathbb{R}^n$ be proper. The degree of $f$ is an integer.
\end{theorem}
\begin{definition}
    An open cover is called a good cover if all nonempty finite intersections
    $U_{p_1} \cap \cdots \cap U_{p_n}$ are diffeomorphic to $\mathbb{R}^n$. A manifold with a finite good cover is said to be of finite type.
    If $\mathcal{U}$, $\mathcal{V}$ are two covers, $\mathcal{U}$ is called the refinement of $\mathcal{V}$
    if every $U_p$ is contained in some $V_q$ and write $\mathcal{V} < \mathcal{U}$.
\end{definition}
\begin{theorem}
    Every manifold has a good cover. If it is compact, the cover can be chosen to be finite.
    Every open cover has a refinement which is a good cover.
\end{theorem}
\begin{corollary}
    The good covers are cofinal in the set of all covers of a manifold(For directed sets, $J \subset I$ is cofinal in $I$ if for every $i\in I$ there is $j\in J$ with $i<j$).
\end{corollary}
\begin{proposition}
    If a manifold has a finite good cover, then its (compact) cohomology is finite dimensional.
\end{proposition}
\begin{lemma}
    The pairing $\left\langle \,,\,\right\rangle : V \otimes  W \rightarrow \mathbb{R}$ for finite vector spaces is nondegenerate iff $v \mapsto\left\langle v,\,  \right\rangle$ is an isomorphism $V\rightarrow W^*$.
\end{lemma}
\begin{lemma}
    Given a commutative diagram of Abelian groups and group homomorphisms
    \begin{center}
    \begin{tikzcd}[cells={nodes={minimum height=2em}}]
        A \arrow[r] \arrow[d, "a"] & B \arrow[r] \arrow[d, "b"] & C \arrow[r] \arrow[d, "c"] & D \arrow[r] \arrow[d, "d"] & E \arrow[d, "e"]\\
        A' \arrow[r] & B' \arrow[r] & C' \arrow[r] & D' \arrow[r] & E'
    \end{tikzcd}
    \end{center}
    where the rows are exact, if $a, b, d, e$ are isomorphisms, so is $c$.
\end{lemma}
\begin{lemma}
    The Mayer-Vietoris sequences may be paired together to form a sign-commutative diagram
    \begin{center}
    \begin{tikzcd}[scale cd=0.8, column sep=0.5em, row sep=2em]
        H^q(U\cup V) \arrow[r] \arrow[d, "\otimes", phantom] & H^q(U) \oplus H^q(V) \arrow[r] \arrow[d, "\otimes", phantom] & H^q(U\cap V) \arrow[r, "d^*"] \arrow[d, "\otimes", phantom] & H^{q+1}(U\cup V) \arrow[d, "\otimes", phantom] \\
        H_c^{n-q}(U\cup V) \arrow[d, "\int_{U\cup V}"] & H_c^{n-q}(U) \oplus H_c^{n-q}(V) \arrow[l] \arrow[d, "\int_U + \int_V"] & H_c^{n-q}(U\cap V) \arrow[l] \arrow[d, "\int_{U\cap V}"] & H^{n-q-1}_c(U\cup V) \arrow[l, "d_*"] \arrow[d, "\int_{U\cup V}"]  \\
        \mathbb{R} & \mathbb{R} & \mathbb{R} & \mathbb{R}                     
    \end{tikzcd}
    \end{center}
    where sign-commutativity means that
    \begin{equation}
        \int_{U\cap V} \omega\wedge d_* \tau = \pm \int_{U \cup V}(d^*\omega)\wedge \tau,
    \end{equation}
    with $\omega \in H^q(U\cap V), \tau \in H_c^{n-q-1}(U\cup V)$.
    This is equivalent to the sign-commutative diagram
    \begin{center}
    \begin{tikzcd}[column sep=0.8em]
        H^q(U\cup V) \arrow[r] \arrow[d] & H^q(U) \oplus H^q(V) \arrow[r] \arrow[d] & H^q(U\cap V) \arrow[d] \\
        H_c^{n-q}(U\cup V)^* \arrow[r] & H_c^{n-q}(U)^* \oplus H_c^{n-q}(V)^* \arrow[r] & H_c^{n-q}(U\cap V)^* \\
    \end{tikzcd}
    \end{center}
\end{lemma}
\begin{theorem}[Poincaré Duality]
    If $M$ is orientable and has a finite good cover, then
    \begin{equation}
        \int H^q(M) \otimes H_c^{n-q}(M) \rightarrow \mathbb{R}
    \end{equation}
    is nondegenerate, or equivalently
    \begin{equation}
        H^q(M) \simeq (H_c^{n-q}(M))^*.
    \end{equation}
\end{theorem}
\begin{corollary}
    If $M$ is connected, oriented with dimension $n$, then $H^n_c(M) \simeq \mathbb{R}$.
    If M is also compact then $H^n(M) \simeq \mathbb{R}$.
\end{corollary}

\begin{definition}
    Let $G$ be a topological group acting effectively(only $1\in G$ acts trivially) on a space $F$ on the left. A surjection $\pi: E \rightarrow B$ between topological spaces is a fiber bundle with fiber $F$
    and structure group $G$(also called a $G$-bundle) if $B$ has an open cover $\{U_a\}$ such that there are fiber-preserving homeomorphisms
    \begin{equation}
        \phi_a : E |_{U_a} \mapsto U_a \times F,
    \end{equation}
    called trivialisations, and continuous transition functions with values in $G$
    \begin{equation}
        g_{ab} = \phi_a\phi_b^{-1} |_{\{x\} \times F} \in G.
    \end{equation}
    $E_x = \pi^{-1}(x)$ is called the fiber at x. $E, B$ are called total and base space respectively.
    If $G$ is not explicitely defined then it is understood to be the group of diffeomorphisms of $F$.
\end{definition}
\begin{remark}
    The transition functions satisfy the cocycle condition
    \begin{equation}
        g_{ab} \cdot g_{bc} = g_{ac}.
    \end{equation}
    Given a cocycle $\{g_{ab}\}$ we can construct a fiber bundle with $\{g_{ab}\}$ as its transition function,
    \begin{equation}
        E=\left(\coprod U_a \times F\right) / (x, y) \sim (x, g_{ab}(x)y).
    \end{equation}
\end{remark}
\begin{theorem}[Künneth Formula]
    For two manifolds $M$, $F$, with at least one having a good finite cover, we have
    \begin{equation}
        H^*(M\times F) = H^*(M) \otimes H^*(F),
    \end{equation}
    which means
    \begin{equation}
        H^n(M\times F) = \oplus_{p+q=n}\, H^p(M) \otimes H^q(F).
    \end{equation}
    Also holds in the compact cohomology case.
\end{theorem}
\begin{theorem}[Leray-Hirsch]
    Let $E$ be a fiber bundle over M with fiber $F$. Suppose $M$ has a finite good cover.
    If there are global cohomology classes $e_1, \dots, e_r$ on $E$ which restricted to each fiber freely generate
    the cohomology of the fiber, then $H^*(E)$ is a free module over $H^*(M)$ with basis $\{e_1, \dots, e_r\}$ meaning that
    \begin{equation}
        H^*(E) \simeq H^*(M) \otimes \mathbb{R}\{e_1, \dots, e_r\} \simeq H^*(M) \otimes H^*(F).
    \end{equation}
\end{theorem}
\begin{definition}
    A (real) vector bundle of rank $n$ is a fiber bundle with fiber $\mathbb{R}^n$ and structure group $GL(n, \mathbb{R})$.
    A section of the vector bundle $E$ over $U \subset B$ open is a map $s: U \rightarrow E$ such that $\pi \circ s$ is the identity on $U$.
    The space of all sections over $U$ is denoted by $\Gamma(U, E)$.
    A frame on $U$ is a collection of sections $s_1, \dots, s_n$ over $U$ such that they form a basis for each $E_x = \pi^{-1}(x),\, x\in U$.
\end{definition}
\begin{lemma}
    If the cocycle $\{g_{ab}'\}$ comes from another trivialization, then there are maps $\lambda_a: U_a \rightarrow GL(n,\mathbb{R})$ such that
    \begin{equation}
        g_{ab} = \lambda_a g_{ab}' \lambda^{-1}_b.
    \end{equation}
    The two cocycles are then said to be equivalent.
\end{lemma}
\begin{proposition}
    Two vector bundles on $M$ are isomorphic iff their cocycles are equivalent.
\end{proposition}
\begin{definition}
    A bundle map is a fiber-preserving smooth map $f: E\rightarrow E'$ of vector bundles which is linear on the fibers.
    If it is possible to find an equivalent cocycle with values in a subgroup $H \subset GL(n, \mathbb{R})$ we say that the structure group of $E$ may be reduced to $H$.
    A vector bundle is orientable if its structure group may be reduced to $GL^+(n,\mathbb{R})$(determinants are positive).
    A trivialisation is said to be oriented if the transition functions have positive determinant.
\end{definition}
\begin{proposition}
    The structure group of a real vector bundle of rank $n$ can be reduced to $O(n)$. It can be reduced to $SO(n)$ iff the bundle is orientable.
\end{proposition}

\begin{definition}
    The direct sum of two vector bundles over $M$ is the direct sum of the local vector spaces and trivialisations,
    likewise for the tensor product. If
    \begin{equation}
        \phi_a : E|_{U_a} \mapsto U_a \times \mathbb{R}^n
    \end{equation}
    is a trivialisation for $E$, then
    \begin{equation}
        (\phi_a^T)^{-1} : E^*|_{U_a} \mapsto U_a \times \left(\mathbb{R}^n\right)^*
    \end{equation}
    is a trivialisation for $E^*$. The transition functions for $E^*$ are then
    \begin{equation}
        (\phi_a^T)^{-1} \phi_b^T = ((\phi_a \phi_b^{-1})^T)^{-1} = (g^T_{ab})^{-1}.
    \end{equation}
\end{definition}
\begin{definition}
    Let $M$ and $N$ be manifolds and $\pi: E \rightarrow M$ a vector bundle over $M$.
    A map $f: N \rightarrow M$ induces a vector bundle $f^{1}E$ on $N$, called the pullback of $E$ by $f$.
    This bundle is defined to be the subset of $N \times E$ given by
    \begin{equation}
        \{ (n, e) \mid  f(n) = \pi(e) \}.
    \end{equation}
    $\text{Vect}_k(M)$ are the isomorphism classes of rank $k$ real vector bundles over $M$.
\end{definition}
\begin{remark}
    The transition functions for $f^{-1}E$ are the pullback functions $f^*g_{ab}$.
    For the composition of two maps $g, f$ between three manifolds we have
    \begin{equation}
        (f \circ g)^{-1} E = g^{-1}(f^{-1}E).
    \end{equation}
\end{remark}
\begin{theorem}[Homotopy Property of Vector Bundles]
    Let $Y$ be a compact manifold. If $f_0$, $f_1$ are homotopic maps from $Y$ to a manifold $X$ and $E$ is a vector bundle on $X$,
    then $f_0^{-1}E$ and $f_1^{-1}E$ are isomorphic.
\end{theorem}
\begin{corollary}
    A vector bundle over a contractible manifold is trivial.
\end{corollary}
\begin{remark}
    Also holds if compact is replaced by paracompact, and also for general topological spaces and continuous maps instead of manifolds and smooth maps.
\end{remark}

\begin{remark}
    Let $E$ be a vector bundle over $M$, then the zero section $x \mapsto (x, 0)$ embeds $M$ diffeomorphically in $E$.
    Since $M \times \{0 \}$ is a deformation retract of $E$, it follows from corollary \ref{homotopyAxiom} that
    \begin{equation}
        H^*(E) \simeq H^*(M).
    \end{equation}
\end{remark}
\begin{lemma}
    An orientable vector bundle over a manifold is an orientable manifold.
\end{lemma}
\begin{proposition}
    If $\pi: E \rightarrow M$ is an orientable vector bundle of rank $n$ and $M$ is orientable of finite type, then $H_c^*(E) \simeq H_c^{*-n}(M)$.
\end{proposition}

\end{document}