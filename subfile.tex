\documentclass[twocolumn]{article}
\usepackage[a4paper, total={7.5in, 10.7in}]{geometry}
\usepackage{amsthm}
\usepackage{enumitem}

\theoremstyle{definition}
\newtheorem{definition}{Definition}[section]
\newtheorem{theorem}[definition]{Theorem}
\newtheorem{lemma}[definition]{Lemma}

\begin{document}

\section{General Topology}
\begin{definition}[Topology]
    A topological space is a set $X$ with a collection of subsets $\mathcal{U}$, called open sets, such that
    \setlist{nolistsep}
    \begin{enumerate}[noitemsep]
        \item $\emptyset, X \in \mathcal{U}$.
        \item The arbitrary union of open sets is open.
        \item The finite union of open sets is open.
    \end{enumerate}
    The complement $X - U$ of an open set $U$ is called closed.
\end{definition}
\begin{definition}
    Let $(X, \mathcal{U})$, $(X, \mathcal{V})$ be topologies.
    $\mathcal{U}$ is called stronger(finer) than $\mathcal{V}$ if $\mathcal{V} \subset \mathcal{U}$,
    and weaker(coarser) if $\mathcal{U} \subset \mathcal{V}$.
\end{definition}
\begin{definition}
    A basis $\mathcal{B}$ of a topology for $X$ is a collection of subsets of $X$ such that
    \setlist{nolistsep}
    \begin{enumerate}[noitemsep]
        \item For each $x \in X$ there is at least one $B \in \mathcal{B}$ with $x \in B$.
        \item If $x \in B_1 \cap B_2$ then there exists a $B_3 \subset B_1 \cap B_2$ with $x \in B_3$.
    \end{enumerate}
    We say that $\mathcal{B}$ generates the topology $\mathcal{U}$ if $U$ is open iff for every $x \in U$ there exits $B \in \mathcal{B}$ with $x \in B \subset U$.
\end{definition}
\begin{lemma}
    Let $\mathcal{B}$ be a basis for a topology $\mathcal{U}$ on $X$.
    Then $\mathcal{U}$ is equal to the collection of all unions of elements of $\mathcal{B}$.
\end{lemma}
\begin{lemma}
    If $\mathcal{C}$ is a collection of open sets, such that for each $U \subset X$ open, $x \in U$
    there is $C \in \mathcal{C}$ such that $x \in C \subset U$, then $\mathcal{C}$ is a basis for X.
\end{lemma}
\begin{lemma}
    Let $\mathcal{B}$, $\mathcal{B}'$ be bases for topologies $\mathcal{U}$, $\mathcal{U}'$ respectively on $X$. Then the following are equivalent:
    \setlist{nolistsep}
    \begin{enumerate}[noitemsep]
            \item $\mathcal{U}'$ is finer than $\mathcal{U}$.
            \item For each $x \in B \in \mathcal{B}$, there is a $B' \in \mathcal{B'}$ with $x \in B' \subset B$.
    \end{enumerate}
\end{lemma}
\begin{definition}
    A subbasis $\mathcal{S}$ for a topology on $X$ is a collection of subsets whose union equals $X$.
    The topology generated by the subbasis is defined to be the collection of all unions of finite intersections of elements of $\mathcal{S}$.
\end{definition}
\begin{definition}
    The product topology on $X \times Y$ is defined by the basis consisting of all sets of the form $U \times V$, $U \subset X$, $V \subset Y$ open.
\end{definition}
\begin{theorem}
    If $\mathcal{B}_1, \mathcal{B}_2$ are bases of $X_1, X_2$ respectively, then
    \begin{equation}
        \mathcal{B} = \{B_1 \times B_2 \, | \, B_1 \in \mathcal{B}_1, B_2 \in \mathcal{B}_2\}
    \end{equation}
    is a basis for $X_1 \times X_2$.
\end{theorem}
\begin{definition}
    The projections are defined by
    \begin{equation}
        \pi_n \, : \, X_1 \times X_2 \rightarrow X_n; \quad \pi_n(x_1, x_2) = x_n.
    \end{equation}
\end{definition}
\begin{theorem}
    The following is a subbasis for $X \times Y$:
    \begin{equation}
        \{ \pi_1^{-1}(U) \, | \, U \in X \,\textrm{open} \}
        \cup \{ \pi_2^{-1}(V) \, | \, V \in Y \,\textrm{open} \}.
    \end{equation}
\end{theorem}
\begin{definition}
    For $(X, \mathcal{U})$ and $Y \subset X$, the subspace topology $\mathcal{V}$ on $Y$ is defined as
    \begin{equation}
        \mathcal{V} = \{ Y \cap U \, | \, U \in \mathcal{U} \}.
    \end{equation}
\end{definition}
\begin{lemma}
    If $\mathcal{B}$ is a basis for $X$ then the following is a basis for $Y$:
    \begin{equation}
        \{ B \cap Y \, | \, B \in \mathcal{B} \}.
    \end{equation}
\end{lemma}
\begin{lemma}
    Let $Y$ be a subspace of X. If $U$ is open in $Y$ and $Y$ is open in X, then $U$ is open in X.
\end{lemma}
\begin{theorem}
    Let $A$, $B$ be subspaces of $X$, $Y$ respectively, then the product topology on $A \times B$ is
    equal the subspace topology $A \times B$ inherits from $X \times Y$.
\end{theorem}
\begin{theorem}
    Arbitrary intersections and finite unions of closed sets are closed.
\end{theorem}
\begin{theorem}
    If $Y$ is a subspace of $X$, then a set is closed in $Y$ iff it equals the intersection of a closed set in $X$ with $Y$.
\end{theorem}
\begin{theorem}
    If $Y$ is a subspace of $X$, $A$ is closed in $Y$ and $Y$ is closed in $X$, then $A$ is closed in $X$.
\end{theorem}
\begin{definition}
    For $A$ a subset of $X$, the closure $\overline{A}$ of $A$ is defined as the intersection of all closed sets containing $A$,
    the interior of $A$ is defined as the union of all sets contained in $A$.
\end{definition}

\end{document}