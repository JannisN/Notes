\documentclass[twocolumn]{article}
\usepackage[a4paper, total={7.5in, 10.7in}]{geometry}
\usepackage{amsthm}
\usepackage{enumitem}

\theoremstyle{definition}
\newtheorem{definition}{Definition}[section]
\newtheorem{theorem}[definition]{Theorem}
\newtheorem{lemma}[definition]{Lemma}
\newtheorem{corollary}[definition]{Corollary}

\begin{document}

\section{General Topology}
\begin{definition}[Topology]
    A topological space is a set $X$ with a collection of subsets $\mathcal{U}$, called open sets, such that
    \setlist{nolistsep}
    \begin{enumerate}[noitemsep]
        \item $\emptyset, X \in \mathcal{U}$.
        \item The arbitrary union of open sets is open.
        \item The finite union of open sets is open.
    \end{enumerate}
    The complement $X - U$ of an open set $U$ is called closed. If $x \in U$ open, then $U$ is called a neighborhood of $x$. Sometimes a non open set $A \supset U$ is also referred to as a neighborhood.
\end{definition}
\begin{definition}
    Let $(X, \mathcal{U})$, $(X, \mathcal{V})$ be topologies.
    $\mathcal{U}$ is called stronger(finer) than $\mathcal{V}$ if $\mathcal{V} \subset \mathcal{U}$,
    and weaker(coarser) if $\mathcal{U} \subset \mathcal{V}$.
\end{definition}
\begin{definition}
    A basis $\mathcal{B}$ of a topology for $X$ is a collection of subsets of $X$ such that
    \setlist{nolistsep}
    \begin{enumerate}[noitemsep]
        \item For each $x \in X$ there is at least one $B \in \mathcal{B}$ with $x \in B$.
        \item If $x \in B_1 \cap B_2$ then there exists a $B_3 \subset B_1 \cap B_2$ with $x \in B_3$.
    \end{enumerate}
    We say that $\mathcal{B}$ generates the topology $\mathcal{U}$ if $U$ is open iff for every $x \in U$ there exits $B \in \mathcal{B}$ with $x \in B \subset U$.
\end{definition}
\begin{lemma}
    Let $\mathcal{B}$ be a basis for a topology $\mathcal{U}$ on $X$.
    Then $\mathcal{U}$ is equal to the collection of all unions of elements of $\mathcal{B}$.
\end{lemma}
\begin{lemma}
    If $\mathcal{C}$ is a collection of open sets, such that for each $U \subset X$ open, $x \in U$
    there is $C \in \mathcal{C}$ such that $x \in C \subset U$, then $\mathcal{C}$ is a basis for X.
\end{lemma}
\begin{lemma}
    Let $\mathcal{B}$, $\mathcal{B}'$ be bases for topologies $\mathcal{U}$, $\mathcal{U}'$ respectively on $X$. Then the following are equivalent:
    \setlist{nolistsep}
    \begin{enumerate}[noitemsep]
            \item $\mathcal{U}'$ is finer than $\mathcal{U}$.
            \item For each $x \in B \in \mathcal{B}$, there is a $B' \in \mathcal{B'}$ with $x \in B' \subset B$.
    \end{enumerate}
\end{lemma}
\begin{definition}
    A subbasis $\mathcal{S}$ for a topology on $X$ is a collection of subsets whose union equals $X$.
    The topology generated by the subbasis is defined to be the collection of all unions of finite intersections of elements of $\mathcal{S}$.
\end{definition}
\begin{definition}
    The product topology on $X \times Y$ is defined by the basis consisting of all sets of the form $U \times V$, $U \subset X$, $V \subset Y$ open.
\end{definition}
\begin{theorem}
    If $\mathcal{B}_1, \mathcal{B}_2$ are bases of $X_1, X_2$ respectively, then
    \begin{equation}
        \mathcal{B} = \{B_1 \times B_2 \, | \, B_1 \in \mathcal{B}_1, B_2 \in \mathcal{B}_2\}
    \end{equation}
    is a basis for $X_1 \times X_2$.
\end{theorem}
\begin{definition}
    The projections are defined by
    \begin{equation}
        \pi_n \, : \, X_1 \times X_2 \rightarrow X_n; \quad \pi_n(x_1, x_2) = x_n.
    \end{equation}
\end{definition}
\begin{theorem}
    The following is a subbasis for $X \times Y$:
    \begin{equation}
        \{ \pi_1^{-1}(U) \, | \, U \in X \,\textrm{open} \}
        \cup \{ \pi_2^{-1}(V) \, | \, V \in Y \,\textrm{open} \}.
    \end{equation}
\end{theorem}
\begin{definition}
    For $(X, \mathcal{U})$ and $Y \subset X$, the subspace topology $\mathcal{V}$ on $Y$ is defined as
    \begin{equation}
        \mathcal{V} = \{ Y \cap U \, | \, U \in \mathcal{U} \}.
    \end{equation}
\end{definition}
\begin{lemma}
    If $\mathcal{B}$ is a basis for $X$ then the following is a basis for $Y$:
    \begin{equation}
        \{ B \cap Y \, | \, B \in \mathcal{B} \}.
    \end{equation}
\end{lemma}
\begin{lemma}
    Let $Y$ be a subspace of X. If $U$ is open in $Y$ and $Y$ is open in X, then $U$ is open in X.
\end{lemma}
\begin{theorem}
    Let $A$, $B$ be subspaces of $X$, $Y$ respectively, then the product topology on $A \times B$ is
    equal the subspace topology $A \times B$ inherits from $X \times Y$.
\end{theorem}
\begin{theorem}
    Arbitrary intersections and finite unions of closed sets are closed.
\end{theorem}
\begin{theorem}
    If $Y$ is a subspace of $X$, then a set is closed in $Y$ iff it equals the intersection of a closed set in $X$ with $Y$.
\end{theorem}
\begin{theorem}
    If $Y$ is a subspace of $X$, $A$ is closed in $Y$ and $Y$ is closed in $X$, then $A$ is closed in $X$.
\end{theorem}
\begin{definition}
    For $A$ a subset of $X$, the closure $\overline{A}_X$ of $A$(in $X$) is defined as the intersection of all closed sets containing $A$,
    the interior of $A$ is defined as the union of all open sets contained in $A$.
\end{definition}
\begin{theorem}
    Let $Y$ be a subspace of $X$, $A$ a subset of $Y$. Then $\overline{A}_Y = \overline{A}_X \cap Y$.
\end{theorem}
\begin{theorem}
    Let $A$ be a subset of $X$.
    \setlist{nolistsep}
    \begin{enumerate}[noitemsep]
        \item Then $x \in \overline{A}$ iff every open set containing $x$ intersects $A$.
        \item If $X$ is given by a basis, then $x \in \overline{A}$ iff every basis element containing x intersects $A$.
    \end{enumerate}
\end{theorem}
\begin{definition}
    $x \in X$ is called a limit point of $A \subset X$ if every neighborhood of $x$ intersects $A$
    in some other point than $x$ itself, or equally if $x$ belongs to the closure of $A-\{x\}$.
\end{definition}
\begin{theorem}
    Let $A'$ be the set of all limit points of $A\subset X$, then
    \begin{equation}
        \overline{A} = A \cup A'.
    \end{equation}
\end{theorem}
\begin{corollary}
    A subset of a topological space is closed iff it contains all its limit points.
\end{corollary}
\begin{definition}
    A sequence $x_k$ converges to $x\in X$ if for each neighborhood $U$ of $x$ there is a positive integer $N$ such that for all $n\leq N$, $x_n \in U$.
\end{definition}
\begin{definition}
    $X$ is called a Hausdorff space if for each $x_1 \neq x_2$ there are neighborhoods of $x_1$ and $x_2$ respectively that are disjoint.
    It is called $T_1$ if all finite point sets are closed.
\end{definition}
\begin{theorem}
    Every finite point set in a Hausdorff space is closed, so it is $T_1$.
\end{theorem}
\begin{theorem}
    Let $X$ be $T_1$, then $x$ is a limit point of $A$ iff every neighborhood of $x$ contains infinitely many points of $A$.
\end{theorem}
\begin{theorem}
    A sequence in a Hausdorff space converges to at most one point. 
\end{theorem}
\begin{theorem}
    The product of two Hausdorff spaces is Hausdorff, so is a subspace of a Hausdorff space.
\end{theorem}

\begin{definition}
    A function $f: X \rightarrow Y$ is called continuous if for every open $V \subset Y$, $f^{-1}(V)$ is open in $X$.
    If $f$ is also a bijection and $f^{-1}$ is also continuous then $f$ is called a homeomorphism.
\end{definition}
\begin{theorem}
    Let $f: X \rightarrow Y$, then the following are equivalent:
    \setlist{nolistsep}
    \begin{enumerate}[noitemsep]
        \item $f$ is continuous.
        \item For every subset $A \subset X$, $f(\overline{A}) \subset \overline{f(A)}$.
        \item For every closed $B \subset Y$, $f^{-1}(B)$ is closed in $X$.
        \item For every $x \in X$ and every neighborhood $V$ of $f(x)$ there is a neighborhood $U$ of $x$ with $f(U) \subset V$.
    \end{enumerate}
    If (4) holds for some point $x \in X$, then $f$ is called continuous at $x$.
\end{theorem}
\begin{definition}
    Let $f: X \rightarrow Y$ be injective and continuous, then $f$ is called an imbedding if $f$ is a homeomorphism under the subspace topology $f(X) \subset Y$.
\end{definition}
\begin{theorem}
    Constant functions, inclusions, composites of continuous functions, restriction of the domain or range of a continuous function to a subspace, are continuous.
    If $f: X \rightarrow Y$ is continuous in the subspace topology when restricted to an open cover $f|U_a$, then $f$ is continuous.
\end{theorem}
\begin{theorem}[The pasting lemma]
    Let $X = A \cup B$, with $A, B$ closed in $X$. Let $f: A \rightarrow Y$ and $g: B \rightarrow Y$ be continuous. If their values agree for every $x \in A \cap B$, then their combination to $X \rightarrow Y$ is continuous.
\end{theorem}
\begin{theorem}
    Let $f: A \rightarrow X \times Y$ be given by $f(a) = (f_1(a), f_2(a))$. Then $f$ is continuous iff $f_1, f_2$ are continuous. Those are called the coordinate functions of $f$.
\end{theorem}

\end{document}