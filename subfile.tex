\documentclass[twocolumn]{article}
\usepackage[a4paper, total={7in, 10in}]{geometry}
\usepackage{amsthm}
\usepackage{enumitem}

\theoremstyle{definition}
\newtheorem{definition}{Definition}[section]
\newtheorem{theorem}[definition]{Theorem}
\newtheorem{lemma}[definition]{Lemma}

\begin{document}

\section{General Topology}
\begin{definition}[Topology]
    A topological space is a set $X$ with a collection of subsets $\mathcal{U}$, called open sets, such that
    \setlist{nolistsep}
    \begin{enumerate}[noitemsep]
        \item $\emptyset, X \in \mathcal{U}$.
        \item The arbitrary union of open sets is open.
        \item The finite union of open sets is open.
    \end{enumerate}
    The complement $X - U$ of an open set $U$ is called closed.
\end{definition}
\begin{definition}
    Let $(X, \mathcal{U})$, $(X, \mathcal{V})$ be topologies.
    $\mathcal{U}$ is called stronger(finer) than $\mathcal{V}$ if $\mathcal{V} \in \mathcal{U}$,
    and weaker(coarser) if $\mathcal{V} \in \mathcal{U}$.
\end{definition}
\begin{definition}
    A basis $\mathcal{B}$ of a topology for $X$ is a collection of subsets of $X$ such that
    \setlist{nolistsep}
    \begin{enumerate}[noitemsep]
        \item For each $x \in X$ there is at least one $B \in \mathcal{B}$ with $x \in B$.
        \item If $x \in B_1 \cap B_2$ then there exists a $B_3 \subset B_1 \cap B_2$ with $x \in B$.
    \end{enumerate}
    We say that $\mathcal{B}$ generates the topology $\mathcal{U}$ if $U$ is open iff for every $x \in U$ there exits $B \in \mathcal{B}$ with $x \in B \subset U$.
\end{definition}
\begin{lemma}
    bla
\end{lemma}

\end{document}