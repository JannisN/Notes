\documentclass[twocolumn]{article}
\usepackage[a4paper, total={7.5in, 10.7in}]{geometry}
\usepackage{amsthm}
\usepackage{amssymb}
\usepackage{amsmath}
\usepackage{enumitem}
\usepackage{tikz-cd}
\tikzcdset{scale cd/.style={every label/.append style={scale=#1},
    cells={nodes={scale=#1}}}}

\theoremstyle{definition}
\newtheorem{definition}{Definition}[section]
\newtheorem{theorem}[definition]{Theorem}
\newtheorem{lemma}[definition]{Lemma}
\newtheorem{proposition}[definition]{Proposition}
\newtheorem{corollary}[definition]{Corollary}
\theoremstyle{remark}
\newtheorem*{remark}{Remark}

\begin{document}

\section{Functional Analysis}

\begin{theorem}[Zorn's lemma]
    Let $X$ be a nonenpty partially ordered set such that every linearly ordered(i.e. $x \prec y$ or $y \prec x$) subset
    has an upper bound in $X$. Then every linearly ordered set has some upper bound that is also maximal in $X$(i.e. for $x$ maximal, $m \prec x$ implies $x = m$ for all $m$).
\end{theorem}
\begin{definition}
    A metric space is a set $M$ with a real-valued function $d(\cdot,\cdot)$ on $M \times M$, called metric, such that
    \setlist{nolistsep}
    \begin{enumerate}[noitemsep]
        \item $d(x, y) \geq 0$
        \item $d(x, y) = 0$ iff $x = y$
        \item $d(x, y) = d(y, x)$
        \item $d(x, z) \leq d(x, y) + d(y, z)$ (triangle inequality)
    \end{enumerate}
    The topology is generated by the base of open balls,
    \begin{equation}
        B_r(x) = \{ y \in M : d(x, y) < r\}.
    \end{equation}
\end{definition}
\begin{definition}
    A sequence $\{x_n\}$ is called Cauchy if
    \begin{equation}
        \forall \epsilon > 0 \, \exists N: N \leq n, m \implies d(x_n, x_m) < \epsilon.
    \end{equation}
\end{definition}
\begin{proposition}
    Any convergent sequence is Cauchy.
\end{proposition}

\begin{definition}
    A complex vector space $V$ is called an inner product space with an inner product
    $(\cdot, \cdot)$ on $V \times V$ such that for all $x, y, z \in V,\, a \in \mathbb{C}$:
    \setlist{nolistsep}
    \begin{enumerate}[noitemsep]
        \item $(x, x) \geq 0$, $(x, x) = 0$ iff $x = 0$
        \item $(x, y + z) = (x, y) + (x, z)$
        \item $(x, ay) = a(x, y)$
        \item $(x, y) = \overline{(y, x)}$
    \end{enumerate}
\end{definition}
\begin{definition}
    Two vectors $x$, $y$ are called orthogonal if $(x, y) = 0$. An orthonormal set is a collection
    of vectors that are mutually orthogonal and such that $\left\lVert x \right\rVert := \sqrt{(x, x)} = 1$ for all $x$.
\end{definition}
\begin{theorem}[Pythagorean]
    Let $\{x_n \}_{n=1}^N$ be orthonormal, then for all $x \in V$
    \begin{equation}
        \left\lVert x\right\rVert^2 = \sum_{n=1}^N \left\lvert (x, x_n)\right\rvert^2 + \left\lVert x - \sum_{n=1}^N (x_n, x)x_n\right\rVert^2.
    \end{equation}
\end{theorem}
\begin{corollary}[Bessel's inequality]
    Let $\{x_n \}_{n=1}^N$ be orthonormal, then for all $x \in V$
    \begin{equation}
        \left\lVert x\right\rVert^2 \geq \sum_{n=1}^N \left\lvert (x, x_n)\right\rvert^2.
    \end{equation}
\end{corollary}
\begin{corollary}[Schwarz inequality]
    \begin{equation}
        \left\lvert (x, y)\right\rvert \leq \left\lVert x\right\rVert \left\lVert y\right\rVert.
    \end{equation}
\end{corollary}
\begin{remark}
    Parallelogram law:
    \begin{equation}
        \left\lVert x + y\right\rVert + \left\lVert x - y\right\rVert = 2 \left\lVert x\right\rVert ^2 + 2 \left\lVert y\right\rVert ^2.
    \end{equation}
\end{remark}
\begin{theorem}
    Every inner product space is a normed linear space with norm $\left\lVert x \right\rVert := \sqrt{(x, x)}$.
\end{theorem}
\begin{definition}
    A complete inner product space is called a Hilbert space, otherwise a pre-Hilbert space.
\end{definition}
\begin{definition}
    Two Hilbert spaces are called isomorphic if there is a linear operator $U$ from $\mathcal{H}_1$ onto(i.e. surjective) $\mathcal{H}_2$, called unitary, such that
    $(Ux, Uy)_{\mathcal{H}_2} = (x, y)_{\mathcal{H}_1}$.
\end{definition}
\begin{lemma}
    Let $\mathcal{M}$ be a closed subspace of $\mathcal{H}$, $x \in \mathcal{H}$.
    Then there is a unique $z \in \mathcal{M}$ closest to $x$.
\end{lemma}
\begin{theorem}[The projection theorem]
    Let $\mathcal{M}$ be a closed subspace of $\mathcal{H}$, $x \in \mathcal{H}$.
    Then $x$ can be uniquely written as $x = z + w$, where $z \in \mathcal{M}$ and $w \in \mathcal{M}^\perp$.
\end{theorem}

\end{document}